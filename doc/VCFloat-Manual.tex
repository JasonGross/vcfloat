\documentclass[article]{memoir}
\usepackage{amssymb}
\usepackage{amsmath}
\usepackage{listings}
\usepackage{lstlangcoq}
\lstset{language=Coq,basicstyle=\sffamily\small,mathescape=true,columns=fullflexible}
\hyphenation{Comp-Cert}
\pagestyle{plain}

%\setstocksize{210mm}{153mm}
\setstocksize{230mm}{178mm}
\settrimmedsize{\stockheight}{\stockwidth}{*}
\settypeblocksize{200mm}{150mm}{*}
\setlrmargins{*}{*}{1}
\setulmargins{*}{*}{1}
\setlength{\footskip}{8mm}
\setlength{\headsep}{5mm}
\setlength{\headheight}{7mm}
\checkandfixthelayout



\title{VCFloat 2.0 Reference Manual}
\author{Andrew W. Appel \and Ariel E. Kellison}

\begin{document}
\maketitle

\chapter{Introduction}
\label{section:Introduction}

VCFloat is a tool for Coq proofs about floating-point round-off error.
When performing a computation such as $x\times 5.7+y$
in floating-point with a fixed number of mantissa bits,
the result of $x\times 5.7$ cannot always be represented exactly
in the same number of bits, ditto the result of the addition $+y$,
so some low-order bits must be thrown away---there is \emph{round-off error}.

We can state this more formally with a bit of notation.
Let \lstinline{%F64} be a notation scope in which
  \lstinline{*} and \lstinline{+} and are interpreted
  as double-precision (64-bit) floating-point operators
  and \lstinline{5.7} is interpreted as a double-precision floating-point constant;
  let $R$ be the function that
  injects a floating-point value into the reals.  Then we might prove
  that
\[
\begin{array}{c}
 1 \le x \le 100 \qquad -1000 \le y \le 1000 \\ \hline
 | \mbox{\lstinline{(x*5.7+y)\%F64}} - (R(x)\times 5.7+R(y)) | \le A
\end{array}
\]
where $A$ is an accuracy bound calculated by VCFloat.

When you prove the correctness and accuracy of a numerical program,
there is far more to do than bound the round-off error.
If we view \lstinline{(x*5.7+y)%F64} as a \emph{floating-point functional model}
  of your program, and $R(x)\times 5.7+R(y)$ as a
  \emph{real-valued functional model} of the same program, then
  the main result of interest can be proved by composing
  these three theorems:
\begin{enumerate}
  \item The real-valued functional model finds a solution to
    the mathematical problem of interest, within accuracy bound $A_1$.
  \item The float-valued functional model approximates the
    real-valued functional model within accuracy bound $A_2$.
  \item The program (in C or Python or whatever) correctly implements
    the float-valued functional model.
\end{enumerate}
VCFloat2 provides
\begin{itemize}
\item A \emph{modelling language} for describing float-valued functional
  models and automatically deriving the corresponding real-valued models;
\item A prover for bounding roundoff error, the difference between the two models;
\item Tools for connecting float-valued models to C programs.  But
  in VCFloat 2.0 (unlike in 1.0) the float-valued modeling language
  is quite independent of C and can be used to reason about numerical
  programs in other languages.
\end{itemize}

\paragraph{To use VCFloat} you follow these steps, we will explain below:
\begin{enumerate}
\item Write down your floating-point functional model as Coq functions on floating-point values.
\item Pick identifiers for your variables and apply a tactic to reify your model.
\item Automatically derive a real-valued functional model.
\item Specify bounds for your input variables, in a \lstinline{boundsmap}.
\item State a roundoff-error theorem, and start the proof using the
  \lstinline{prove_rndval} tactic.
\item Prove the stage-1 verification conditions; usually this is as easy as
  writing \lstinline{all:interval}.
\item Finish proving the main theorem; sometimes this is completely automatic, sometimes you have to assist.
\item (optional) Prove that your C program correctly implements the functional model.
\item (optional) Prove that your real-valued model accurately approximates the
  mathematical quantity of interest.
\end{enumerate}

\chapter{Floating-point functional models}
To use VCFloat you start with,
\begin{lstlisting}
Require Import vcfloat.VCFloat.
\end{lstlisting}
This imports VCFloat's functional modelling language and all of its
provers.  

Functional models are written as expressions in Coq that apply
functions (such as \emph{add} and \emph{multiply}) to
variables, constants, and subexpressions that belong to floating-point
\emph{types}.  We will start with the types.

\begin{lstlisting}
type : Type $\mbox{\quad\qquad\qquad\emph{each floating-point format is described as a}~\textsf{type}}$
ftype: type -> Type $\mbox{\qquad\emph{a floating-point number in format \(\mathsf{t}\) belongs to Coq type} \textsf{ftype(t)}}$
TYPE : forall (precp : positive) (femax: Z), (fprecp<femax) -> (1<fprecp) -> type.
Tsingle: type := TYPE 24 128 I I.
Tdouble: type := TYPE 53 1024 I I.
\end{lstlisting}

That is, you specify a floating-point format, a \lstinline{type}, by
the number of mantissa bits (e.g., 24 for single-precision, 53 for
double-precision, but any number $\ge 2$ is legal) and a maximum
exponent value (128 for single-precision, 1024 for double-precision,
any number greater than the number of mantissa bits).
\lstinline{TYPE} is a constructor for \lstinline{type}, and the
\lstinline{I} arguments happen to be proofs that \lstinline{24<128}
and \lstinline{1<24}, and so on.

\chapter{NaNs}
In the IEEE-754 floating-point standard, one cannot
simply \emph{add} two numbers, one must specify how the NaNs will
be propagated.  That is, if $x$ and $y$ are double-precision floats,
what Not-a-Number (NaN) should float-add return if $x$ or $y$ or both are
Not-a-Number?  Unfortunately that is left to each computer architecture 
to decide.  VCFloat wants to be rigorously faithful to the
semantics of the actual computation, so we specify the
NaN-propagation behavior of the floating-point model in a
typeclass \lstinline{Nans}.

The good news is that if your computation never produces any
NaNs, then it won't matter which instance of the \lstinline{Nans}
typeclass you use.  And VCFloat helps you prove that your computation
never produces NaNs.
Then you can parameterize your float-functional-model
over a \lstinline{NANS} parameter
 as follows:

\begin{lstlisting}
Section WITHNANS.
Context {NANS: Nans}.

$\mbox{\emph{\ldots your functional model goes here }}$
End WITHNANS.
\end{lstlisting}
That is, the \lstinline{NANS} variable can be instantiated with
\emph{any} architecture-specific which-nans-to-use structure,
and then your float-functional-model will consult this NANS
structure whenever it produces a not-a-number, which you will prove
is never.

\chapter{Notation Scopes}
These notation scopes (and their delimiters) come with VCFloat:
\begin{lstlisting}
Delimit Scope float32_scope with F32.
Delimit Scope float64_scope with F64.
\end{lstlisting}

Delimiters \%F32 and \%F64 indicate that constants and operator-symbols should stand for single precision and double-precision (respectively) values and functions.
\begin{lstlisting}
Definition myformula ($h$: ftype Tdouble) := (5.0e-1 + cast Tsingle ($h$ * 1.6)%F64)%F32.
\end{lstlisting}
Here, the constant 1.6 and operator \lstinline{*} are
interpreted in double precision,
and the constant 5.0e-1 (which could just as well have been written
as .5) and operator \lstinline{+} are interpreted in single precision.
The variable $h$ is a double-precision floating-point number.

\chapter{Operators}
The following operators are available in each notation scope:
\begin{lstlisting}
  + $~$ - $~$ * $~$ / $~$ < $~$ <= $~$ > $~$ >= 
\end{lstlisting}
The minus sign \lstinline{-} can be used infix (subtraction) or prefix (negation).  The comparison operators can be used in the style
\lstinline{$x$ <= $y$ < $z$} as usual in Coq.
The following functions can also be used:
\begin{description}
\item[\textsf{BABS}]  ~~~(absolute value)
\item[\textsf{BSQRT}]  ~~~(square root)
\item[\textsf{cast~$t$}]  ~~~~(cast to $\mathsf{ftype}(t)$)
\end{description}

\chapter{Example}

A mass on a spring---a harmonic oscillator---with position $x$ and
velocity $v$ can be simulated over time-step $h=\frac{1}{32}$ using the
Verlet (``leapfrog'') method with the formula,

\begin{lstlisting}
Definition h := (1/32)%F32.
Definition F(x: ftype Tsingle) : ftype Tsingle := (3.0-x)%F32.  
Definition step (x v: ftype Tsingle) := (x + h*(v+(h/2)*F(x)))%F32.
\end{lstlisting}

Here, the function \lstinline{step} is the functional model of
(part of) a C program:

\begin{lstlisting}[language=C]
const float h = 1.0/32.0;
float F (float x) { return 3.0f-x; }
float step (float x, float v) { return x+h*(v+h/2.0f)*F(x); }
\end{lstlisting}

\chapter{Reification}

VCFloat will \emph{reify} your functional model into the internal
syntax tree that it uses.  In order to do this, it will need
a \emph{name} for each of your variables.
VCFloat's name type is the Coq positive numbers.
In our example the variables are $x$ and $v$, and
we will use 1 and 2 for their names:

\begin{lstlisting}
Definition _x : ident := 1%positive.  (* Variable name for position *)
Definition _v : ident := 2%positive.  (* Variable name for velocity *)
\end{lstlisting}

Here, the Coq variable \lstinline{_x} contains not the value,
but the \emph{identifier} that we will use for the floating-point
variable $x$.  It is not necessary to use consecutive positives,
we could have used 5 and 2.
Now we can connect \lstinline{_x} and
\lstinline{_v} to $x$ and $v$ as follows:

\begin{lstlisting}
Definition step' := ltac:(let e' := HO_reify_float_expr constr:([_x; _v]) step in exact e').
\end{lstlisting}

This is a tactical definition of a VCFloat abstract-syntax tree,
\lstinline{step'}, the reified version of \lstinline{step}.
The tactic is called \lstinline{HO_reify_float_expr},
and it expects its second argument (in this case, \lstinline{step})
to be a function from (zero or more) floating-point values
to a floating-point value.  It learns how many arguments
there should be from examining the Coq type of \lstinline{step}.
In this case, since \lstinline{step} has type
\lstinline{ftype Tsingle -> ftype Tsingle -> ftype Tsingle},
the tactic knows that \lstinline{step} should have
two arguments, both single-precision floats.

The first argument of \lstinline{HO_reify_float_expr}
should be list of identifiers, to associate with those
parameters of the functional model.  In this case the list is
simpliy \lstinline{[_x;_v]}.

\chapter{Boundsmap}

In order to do round-off analysis one generally needs \emph{bounds}
in the input variables:
For example, what are the lowest and highest possible
values of $x$ and $v$ in our example?
We gather information about each variable
(name, floating-point type, low-bound, high-bound) into
a \lstinline{boundsmap}, which maps variable-identifiers to
\lstinline{varinfo} structures.
\begin{lstlisting}
Record varinfo := {var_type: type; var_name: ident; var_lobound: R; var_hibound: R}.
\end{lstlisting}

To create the \lstinline{boundsmap}
first make a list of \lstinline{varinfo}s,
then use some \lstinline{ltac} boilerplate to compute.
\begin{lstlisting}
Definition step_bmap_list : list varinfo := 
  [ Build_varinfo Tsingle _x 2 4 ;  Build_varinfo Tsingle _v (-2)  2 ].

Definition step_bmap : boundsmap :=
   ltac:(let z := compute_PTree (boundsmap_of_list $\mathit{step\_bmap\_list}$) in exact z).
\end{lstlisting}
In the first definition, we make a list of \lstinline{varinfo} structures.
For each parameter of the functional model, we specify its floating-point precision, its identifier, its lowest possible input value,
and its highest possible input value.  
We put these into a list---in our example, \lstinline{step_bmap_list}.
Then the tactical definition (\lstinline{step_bmap}) is
a line of boilerplate that always looks the same
(except for the italicized part where you specify this list
as shown above).


\chapter{Valmap and reflection}
\label{valmap}

You can \emph{reflect} the abstract-syntax tree (such as \lstinline{step'})
back into a functional model (such as \lstinline{step}).
To do that, first make a \lstinline{valmap} that relates your
variable identifiers to floating-point values.

\begin{lstlisting}
Definition step_vmap_list ($x$ $v$ : ftype Tsingle) := [(_x, existT ftype _ $x$);(_v, existT ftype _ $v$)].

Definition step_vmap (x v : ftype Tsingle) : valmap :=
 ltac:(let z := compute_PTree (valmap_of_list (step_vmap_list x v)) in exact z).
\end{lstlisting}
The auxiliary definition \lstinline{step_vmap_list}
(when applied to $x$ and $v$)
is a list of pairs, identifier$\times$value,
where the ``value'' is a dependent pair of a \lstinline{type} (a floating
point format such as Tsingle or Tdouble) and a value of that type.
In this case, both $x$ and $v$ are single-precision,
but valmaps have the ability to mix precisions.

The second step computes this association list into
an efficient data structure.

The function \lstinline{fval} evaluates the floating-point interpretation
of an AST, in an environment that maps the variables.
To \emph{reflect} an AST using a valmap, apply \lstinline{fval} as follows:

\begin{lstlisting}
Definition reflected_step (x v: ftype Tsingle) :=  fval (env_ (step_vmap x v)) step'.

Lemma reflect_reify : forall x v, reflected_step x v = step x v.
Proof. reflexivity. Qed.
\end{lstlisting}
The lemma demonstrates that the round-trip---reify then reflect---is indeed
the identity function.

\chapter{Real-valued functional model}

Suppose we take the float-valued functional model (the \lstinline{step}) 
function) and interpret every constant and operator in the real numbers:

\begin{lstlisting}
Definition step_realmodel' (x v: ftype Tsingle) : R := FT2R x + (1/32)*(FT2R v + ((1/32)/2)*(3- FT2R x)).
\end{lstlisting}
You can make this look prettier using a coercion:

\begin{lstlisting}
Coercion FT2R: ftype >$ $-$ $> R.

Definition step_realmodel (x v: ftype Tsingle) : R :=  x + (1/32)*(v + ((1/32)/2)*(3-x)).
\end{lstlisting}

In fact, you can automatically derive a real-valued functional model
using the \lstinline{rval} function, which reflects into the reals
much like \lstinline{fval} reflects into the floats.  Here's
a theorem showing that you get what you'd expect:

\begin{lstlisting}
Lemma correspond_floatmodel_realmodel: forall x v, rval (env_ (step_vmap x v)) step' = step_realmodel x v.
Proof. intros. unfold step_realmodel. simpl. repeat f_equal; compute; lra. Qed. 
\end{lstlisting}

\chapter{Round-off theorem}
The purpose of VCFloat is to prove how accurately the
float-valued functional model approximates the real-valued
functional model.  Here's an example of such a theorem:
 
\begin{lstlisting}
Lemma prove_roundoff_bound_step:  forall $\mathit{vmap}$, prove_roundoff_bound step_bmap $\mathit{vmap}$ step'  (/ 4000000).
\end{lstlisting}
This says, for any valmap \emph{vmap}
containing values for $x$ and $v$
that are within the bounds specified by \lstinline{step_bmap},
the difference between the floating-point
interpretation of \lstinline{step'}
and the real-number interpretation of \lstinline{step'}
will be less than one four-millionth.

Recall, of course, that ``the floating-point interpretation of \lstinline{step'}'' is exactly our float-valued functional model;
and ``the real-number interpretation of \lstinline{step'}''
is exactly our real-valued functional model.

What if we didn't know the accuracy 1/400000 in advance?  See the
next chapter.

Here is how we prove the theorem:
\begin{lstlisting}
Lemma prove_roundoff_bound_step:  forall $\mathit{vmap}$, prove_roundoff_bound step_bmap $\mathit{vmap}$ step'  (/ 4000000).
Proof.
intros.
prove_roundoff_bound.
-
 prove_rndval.
 all: interval.
- 
 prove_roundoff_bound2.
 prune_terms (cutoff 30).
 do_interval.
Qed.
\end{lstlisting}
Step one, \lstinline{prove_roundoff_bound}, is always the same,
and leaves two subgoals (delimited by ``bullets'').
The first subgoal is always proved with \lstinline{prove_rndval},
which leaves a few verification conditions.
In this case, there are three: proving that the additions
and subtractions do not overflow.  In general the subgoals
left by \lstinline{prove_rndval} are easy to prove using the Coq
Interval package, as shown here by \lstinline{all:interval}.

The second subgoal is always proved by
\lstinline{prove_roundoff_bound2}, which leaves
one subgoal.  In this case the subgoal is,

\begin{lstlisting}
NANS : Nans
v_v : R, $\qquad$ BOUND : -2 $\le$ v_v $\le$ 2
v_x : R, $\qquad$ BOUND0 : 2 $\le$ v_x $\le$ 4
e0 : R, $\qquad$ E : Rabs e0 $\le$ powerRZ 2 (-150)
d : R, $\qquad$ E0 : Rabs d $\le$ powerRZ 2 (-24)
e1 : R, $\qquad$ E1 : Rabs e1 $\le$ powerRZ 2 (-150)
e2 : R, $\qquad$ E2 : Rabs e2 $\le$ powerRZ 2 (-150)
d0 : R, $\qquad$ E3 : Rabs d0 $\le$ powerRZ 2 (-24)
e3 : R, $\qquad$ E4 : Rabs e3 $\le$ powerRZ 2 (-150)
e4 : R, $\qquad$ E5 : Rabs e4 $\le$ powerRZ 2 (-150)
d1 : R, $\qquad$ E6 : Rabs d1 $\le$ powerRZ 2 (-24)
______________________________________(1/1)
Rabs ((v_x + (1/32 * ((v_v + (1/64 * ((3-v_x)*(1+d0)+e1) + e4)) * (1+d) + e3) + e2)) * (1+d1) + e0
      - (v_x + 1/32 * (v_v + 1/32 / 2 * (3-v_x))))
 $\le$  / 4000000
\end{lstlisting}
\label{stage2proofgoal}
That is, the real-valued variables \lstinline{v_x} and \lstinline{v_v},
which represent the values of $x$ and $v$, are within the bounds
specified in the boundsmap.  The variables 
$\delta,\delta_0,\delta_1$ that represent
relative errors of additions and subtractions, are each less than
$2^{-24}$ in absolute value.  The variables 
$\epsilon,\epsilon_0,\epsilon_1,\epsilon_2,\epsilon_3,\epsilon_4$ that
represent absolute errors of additions and multiplications
are each less than $2^{-150}$ in absolute value.
Finally, assuming all of that, one must prove that the
difference between the computation \emph{with} all the
deltas and epsilons and the computation \emph{without}
the deltas and epsilons, is less than the accuracy bound.

To prove this, one can use the Coq Interval package.  But in many
cases one must do some work to prepare the goal for solution by
Interval.  Later chapters will explain.
In this example, we use the \lstinline{prune_terms} tactic.
Then the goal solves by \lstinline{do_interval}.

\chapter{Letting VCFloat calculate the accuracy bound}

You can ask VCFloat not only to \emph{prove} an accuracy bound,
but to \emph{calculate and prove} the bound if you don't know it
in advance.

\begin{lstlisting}
Definition find_and_prove_roundoff_bound (bmap: boundsmap) (e: expr) :=
  {bound: R | forall vmap, prove_roundoff_bound bmap vmap e bound}.
\end{lstlisting}

This useful definition expresses a dependent pair of a real-valued
\lstinline{bound} with a proof that this really is a bound
on the accuracy of a floating-point program.  To use this with
our ``step'' example, one could state this theorem:

\begin{lstlisting}
Lemma find_and_prove_roundoff_bound_step :
  find_and_prove_roundoff_bound step_bmap step'.
Proof.
eexi$ $sts.
intro.
prove_roundoff_bound.
-
 prove_rndval; interval.
-
 prove_roundoff_bound2.
 prune_terms (cutoff 30).
 do_interval.
Defined.
\end{lstlisting}

Within the proof, the only differences are,
\begin{itemize}
\item  The \lstinline{eex$$ists} tactic introduces a unification variable
  for the accuracy parameter.
\item The proof ends with \lstinline{Defined} instead of \lstinline{Qed}
  so that when it's done, one can extract the actual value of the bound.
\end{itemize}

\chapter{field\_simplify}

The subgoal remaining after \lstinline{prove_roundoff_bound2} has the
form, $|E'-E|\le A$, where $E$ is the real-valued functional model,
$E'$ is the real-valued model with deltas and epsilons inserted to
represent round-off errors, and $A$ is the accuracy bound.
Goals of this kind can be given to the Coq Interval package to solve.
Interval works by computing in interval arithmetic,
where every number is represented by a pair of floating-point numbers
representing a lower bound and upper bound.  For example, to subtract
$a-b$ represented as $(a_\mathrm{lo},a_\mathrm{hi})-(b_\mathrm{lo},b_\mathrm{hi})$,
the result is $(a_\mathrm{lo}-b_\mathrm{hi},~a_\mathrm{hi}-b_\mathrm{lo})$.
For soundness Interval even takes care to set the floating-point rounding modes
to round down when computing the lo-bound, and round-up when computing
the high bound.

When variables such as \lstinline{v_x} and \lstinline{v_v} can
take on a wide range of possible values, Interval uses
repeated \emph{bisection} to measure many subranges
of \lstinline{v_x} and \lstinline{v_v}, taking the maximum error.

But there's a problem.  Consider the interval calculation of
$(a+\delta)-a$, which comes out to
\[(a_\mathrm{lo}+\delta_\mathrm{lo}-a_\mathrm{hi},~~a_\mathrm{hi}+\delta_\mathrm{lo}-a_\mathrm{lo}).\]
If the value of $a$ is only approximately known, so
$a_\mathrm{hi}-a_\mathrm{lo}$ is large, then the interval approximation
of $(a+\delta)-a$ is similarly large---which is bad.  But we know that whatever the \emph{true} value of $a$ is, when subtracted from itself it will yield zero.
That is, a much better approximation can be obtained by
\emph{symbolically} subtracting $a+\delta-a=\delta$
before calling the interval package.

If you examine the proof goal at the end of section~\ref{stage2proofgoal},
you'll see that it contains (more or less)
$\mathsf{v\_x}(1+\delta)-\mathsf{v\_x}$ and
$\mathsf{v\_v}(1+\delta)-\mathsf{v\_v}$.  So we can expect the Interval
tactic to perform badly on this expression.
The solution is to symbolically simplify the expression,
and Coq's \lstinline{field_simplify} tactic can do that:

\begin{lstlisting}
match goal with |- Rabs ?a <= _ $~~$   => field_simplify a end.
\end{lstlisting}
This changes the below-the-line portion of the proof goal to,
\begin{lstlisting}
Rabs((-v_x*d0*d*d1-v_x*d0*d-v_x*d0*d1-v_x*d0-v_x*d*d1-v_x*d+2047*v_x*d1+64*v_v*d*d1+
   64*v_v*d+64*v_v*d1+3*d0*d*d1+3*d0*d+3*d0*d1+3*d0+e1*d*d1+e1*d+e1*d1+e1+64*e4*d*d1+
   64*e4*d+64*e4*d1+64*e4+3*d*d1+3*d+64*e3*d1+64*e3+2048*e2*d1+2048*e2+3*d1+2048*e0)/2048)
 <= / 4000000
\end{lstlisting}
Applying the \lstinline{interval} tactic solves this goal immediately.

But \lstinline{field_simplify} is not the best tool for this job;
we use it here only to illustrate the principle of (automatically) expanding the
formula into a multinomial and symbolically canceling terms.
There are two problems: (1) the multinomial can have an exponential number of terms, most of which are negligible; and (2) floating-point functional models do not always expand into nice multinomials.

\chapter{prune\_terms}

The proof goal at at the end of the last section had many terms
similar to \lstinline{3*d0*d*d1} where two or more deltas or epsilons
are multiplied together.  Since the deltas are bounded by $2^{-24}$
(in single precision) or $2^{-53}$ (in double precision),
and the epsilons are much smaller than that, their product is
probably negligible.

The \lstinline{prune_terms} tactic expands a formula into a
multinomial (like \lstinline{field_simplify}) but also
deletes (and bounds) negligible terms---you specify
a ``cutoff'' for what you consider neglible.  For example,
at the section-\ref{stage2proofgoal} proof goal,
one can write,

\begin{lstlisting}
prune_terms (cutoff 30).
\end{lstlisting}
which expands into a multinomial, cancels terms symbolically,
and deletes all terms that can be bounded by $2^{-30}$.
The result is,

\begin{lstlisting}
Rabs (1 * d1 * v_x + 1/32 * d * v_v + 1/32 * d1 * v_v)
 <=  / 4000000 - 5910977010729000 / 9671406556917033397649408
\end{lstlisting}
In this goal, the number \lstinline{5910977010729000 / 9671406556917033397649408} is the sum of the bounds of the negligible terms.
The goal solves easily by the \lstinline{interval} tactic.

For comparison, using \lstinline{(cutoff 50)} gives the goal,
\begin{lstlisting}
Rabs (-1/2048 * d0 * v_x + -1/2048 * d * v_x + 2047/2048 * d1 * v_x +
      1/32 * d * v_v + 1/32 * d1 * v_v + 3/2048 * d0 + 3/2048 * d + 3/2048 * d1)
   <= / 4000000 - 5242471455916076 / 20282409603651670423947251286016
\end{lstlisting}
in which not as many terms have been neglected.  But either way,
\lstinline{interval} solves the goal.
\chapter{error\_rewrites} Suppose an expression for the absolute
forward error does not expand into a nice multinomial
that is tractable for the \lstinline{prune_terms} tactic;
suppose it such a large expression
that applying the \lstinline{field_simplify+interval} 
tactic causes Coq to crash. An example
of such a problem is the \lstinline{carbonGas} benchmark from the
FPBench benchmark suite:
\begin{lstlisting} 
carbonGas($v$):= P+A*(N/$v$)*(N/$v$)*($v$-N*B)-K*N*T
\end{lstlisting}
where \lstinline{A,B,K,N,P,T} are constants. Because $v$ 
appears in the denominator of terms in \lstinline{carbonGas},
VCFloat's \lstinline{prune_terms} tactic won't simplify the 
expression enough for \lstinline{interval} to produce a decent 
bound. A tactic that can be used in cases like this is 
\lstinline{error_rewrites}, which will recursively
decompose a main proof goal for absolute forward error into 
subgoals of smaller subexpressions on related terms using the 
following equalities.  
 
\begin{align*}
 (\tilde{u} - \tilde{v}) - (u-v) &= (\tilde{u} - u) -
(\tilde{v} - v) \\
 (\tilde{u} + \tilde{v}) - (u+v) &= (\tilde{u} - u) +
(\tilde{v} - v) \\ 
(\tilde{u} * \tilde{v}) - (u*v) &= (\tilde{u} - u)*v +
(\tilde{v} - v)*u + (\tilde{u} - u)*(\tilde{v} - v) \\
\frac{u'}{v'} - \frac{u}{v} &= (u'-u) - (v'-v) * \frac{1}{v} * 
u
\end{align*}
We write $\tilde{e}$ to denote an expression with 
deltas (for relative error) and epsilons (for absolute error); 
so $\tilde{u} - u$ is just the absolute error in computing the 
formula $u$ in floating-point (rather than in the real numbers).

Consider the expression for the absolute
forward error of \lstinline{carbonGas}, which looks like 
\begin{align*}
 \texttt{Rabs}((\tilde{u}*(1+\delta_2) + \epsilon_7 - 
 \tilde{v}) * (1 + \delta_6) + \epsilon_0 - (u - v)).
\end{align*}
Applying \lstinline{error_rewrites}
produces -- supposing that $\tilde{u}$, $\tilde{v}$, $u$, and 
$v$ are 
opaque -- three subgoals:
\paragraph{subgoal 1:}$\texttt{Rabs} (\tilde{u}*(1+\delta_2)*(1 
+ \delta_6) + 
\epsilon_7 -  u) \le \ ?e3$

\paragraph{subgoal 2:}
$\texttt{Rabs}(\tilde{v} * (1 + \delta_6) - v)  \le \ ?e2$

\paragraph{subgoal 3:}
$\texttt{Rabs}(\epsilon_0) \le \ ?e1$

\noindent where $?e1,?e2,?e3$ are unification variables
to be determined in subproofs, and
the total error is now bounded by $e3 + e2 + e1$. 

The subexpressions in the above subgoals are smaller 
and 
contain related terms. For these reasons, using the 
\lstinline{field_simplify+interval} tactic is potentially more 
tractable. However, in 
some cases, the recursion stops before it fully decomposes a 
subexpression into a form that \lstinline{interval} can provide 
a decent bound on. In this case, as long as all rational 
expressions have been decently decomposed by the division case 
in \lstinline{error_rewrites}, the \lstinline{prune_terms} tactic can 
be used successfully. 

There are some cases when \lstinline{error_rewrites} causes Coq 
to crash: on expressions with a large number of operations 
(approximately 40). This occurs because 
\lstinline{error_rewrites} simply produces too many subgoals. 
As an example, consider that the 
\lstinline{carbonGas} benchmark has 11 floating-point 
operations and that \lstinline{error_rewrites} produces 125 
subgoals (and this is even after \lstinline{error_rewrites} has 
automatically discharged a few!). 

\chapter{Abstract \emph{versus} transparent valmap}

Many simple applications of VCFloat prove a theorem,
``Given \lstinline{boundsmap} $\mathit{bm}$,
for any \lstinline{valmap} $\mathit{vm}$ that satisfies the
low-bound/high-bound constraints in $\mathit{bm}$,
the round-off error will be less than $A$.''
In such a case it works well to quantify over $\mathit{vm}$,
a variable representing a valmap.

But sometimes you want to prove an accuracy bound that's
a function of the input variables (such as a relative error bound).
Or you may have have additional constraints on the variables
in $\mathit{vm}$.
In such cases you can use an advanced technique:
quantify over the individual variables, with additional constraints
expressed directly in Coq as needed, and then state a
\lstinline{prove_roundoff_bound} theorem on a transparently
constructed valmap.  We illustrate here (and see Test/summation.v
for a full demonstration).

Section~\ref{valmap} showed how to construct \lstinline{step_valmap}
from the individual variables $x$ and $v$.  You can leverage this
to state a roundoff theorem by quantifying over $x$ and $v$:
\begin{lstlisting}
Lemma prove_rndoff' :
  forall (x v : ftype Tsingle),
  let accuracy := $\mbox{\emph{some function of x and v}}$
  my_extra_constraint x v ->   
  prove_roundoff_bound step_bmap (step_vmap x v) step' accuracy.
\end{lstlisting}  

\chapter{Annotations}

Floating-point error analysis can be slightly more precise
in certain cases:

\begin{description}
\item[Denorm:]  When the result of a calculation is known to be a
  \emph{denormal} (also called \emph{subnormal}) number---a tiny
  number within $2^{e_\mathrm{min}}$ of zero---then it has only
  an additive error.  That is, $(a+b)+\epsilon$ instead of
  $(a+b)(1+\delta)+\epsilon$.
\item[Norm:]
   When the result of a calculation is known to be a
   \emph{normal} number---that is, bounded away from zero by
   at least $2^{e_\mathrm{min}}$---then it has only
  a relative error.  That is, $(a+b)(1+\delta)$ instead of
  $(a+b)(1+\delta)+\epsilon$.
\item[Sterbenz:]  When $a,b$ satisfy $\frac{1}{2} < \frac{a}{b} < 2$,
  then the floating point subtraction $a-b$ is exact, no relative
  error $\delta$, no absolute error $\epsilon$.
\end{description}

You can annotate these cases in your functional model using these functions:

\begin{lstlisting}
Definition Norm {A}(x: A) := x.
Definition Denorm {A}(x: A) := x.
Definition Sterbenz {A}(x: A) := x.
\end{lstlisting}

As you can see, these are just identity functions, so semantically they
do nothing.  But they guide VCFloat's reifier to mark its internal
abstract-syntax tree.  This will cause additional proof obligations
(subgoals) at stage 1, to \emph{prove} that such-and-such a subexpression
is normal, or denormal, or Sterbenz; but will cause fewer
deltas and epsilons to be generated at stage 2.

In our running example we could write,
\begin{lstlisting}
Definition h := (1/32)%F32.
Definition F(x: ftype Tsingle) : ftype Tsingle := Sterbenz(3.0-x)%F32.  
Definition step (x v: ftype Tsingle) := Norm(x + h*(v+(h/2)*F(x)))%F32.
\end{lstlisting}

\chapter{Verified Software Toolchain}

You can use the Verified Software Toolchain (VST) to prove that a C program
correctly implements a floating-point functional model.

Along with importing \lstinline{VST.floyd.proofauto} and
the other standard boilerplate that introduces a VST proof,
you will want:

\begin{lstlisting}
From vcfloat Require Import FPCompCert Float_notations.
Require Import float_model.  (* your functional model *)
\end{lstlisting}

It is not necessary to import all of vcfloat.VCFloat; the imports shown
are enough to connect CompCert's definitions for floating point
to VCFloat's definitions.  CompCert and VCFloat use different names
for the same underlying Flocq floating-point types:

\begin{lstlisting}
Eval compute in compcert.lib.Floats.float32.   (* = Binary.binary_float 24 128 *)
Eval compute in ftype Tsingle.   $~~~~~~~~~~~~~~~$              (* = Binary.binary_float 24 128 *)
Eval compute in compcert.lib.Floats.float.  $~$   (* = Binary.binary_float 53 1024 *)
Eval compute in ftype Tdouble.   $~~~~~~~~~~~~~$              (* = Binary.binary_float 53 1024 *)
\end{lstlisting}

In your VST assertions (funspecs, loop invariants,
etc.), use the VCFloat names for those types.
For example, here we write \lstinline{ftype Tsingle} instead of
\lstinline{float32}:

\begin{lstlisting}
Definition force_spec :=
 DECLARE _force
 WITH  q : ftype Tsingle
 PRE [ tfloat ] PROP() PARAMS(Vsingle q) SEP()
 POST [ tfloat ] PROP() RETURN (Vsingle (F q)) SEP().
\end{lstlisting}

There are two useful tactics to convert CompCert-style float notations to VCFloat-style notations.  Immediately after \lstinline{start_function} you
can write,

\begin{lstlisting}
start_function.
subst MORE_COMMANDS; unfold abbreviate; canonicalize_float_constants.
\end{lstlisting}

The tactic \lstinline{canonicalize_float_constants} converts
all of the floating-point literals in the AST of your function-body
into a VCFloat style, which makes them easier to reason about
(and to relate to your functional model).

The following tactic is useful \emph{after} going \lstinline{forward}
through a sequence of C statements that perform floating-point
operations:
\begin{lstlisting}
autorewrite with float_elim in *.
\end{lstlisting}
It converts \lstinline{Float32.add $x$ $y$} to \lstinline{($x$+$y$)%F32},
  and similarly for other operators.

That's it!  Other than these conversions, you use VST in
a completely standard way.  

\subsection*{Exactly matching the functional model}

When using VST or any other tool to prove that a program correctly
implements a functional model, take care to
match it exactly.  For example, in floating point the
associative law $(a+b)+c=a+(b+c)$ does not hold.

The commutative law $a+b=b+a$ may hold \emph{only if
neither $a$ nor $b$ is a NaN}, depending on how your target
machine propagates NaNs.  Therefore, if your C program
computes $a+b$ while the functional model computes $b+a$,
you will be able to prove it correct only if you propagate
the invariant that $a$ is finite and $b$ is finite.
It's certainly possible to propagate such invariants, but
it's simpler if you don't have to.

\chapter{Bibliography}

\quad VCFloat 1.0 was built in 2015 and described in,\newline
\textbf{A unified Coq framework for verifying C programs with floating-point computations}, by Tahina Ramananandro, Paul Mountcastle, Beno\^{\i}t  Meister, and Richard Lethin, in \emph{Proceedings of the 5th ACM SIGPLAN Conference on Certified Programs and Proofs (CPP'16)}, pages 15--26, 2016\newline (https://doi.org/10.1145/2854065.2854066).
\vspace\baselineskip

VCFloat 2.0 was built 2021-2022 and described in,\newline
\textbf{VCFloat2: Floating-point error analysis in Coq},
by Andrew W. Appel and Ariel E. Kellison,
October 2022  (distributed as doc/vcfloat2.pdf in the vcfloat repo).
\vspace\baselineskip

VCFloat 2.0 is applied and demonstrated in,\newline
\textbf{Verified numerical methods for ordinary differential equations},
by Ariel E. Kellison and Andrew W. Appel,
in \emph{15th International Workshop on Numerical Software Verification (NSV'22)}, August 2022.



\end{document}
